\documentclass[spanish, c]{beamer}

\usepackage[utf8]{inputenc}
%\usepackage[spanish, mexico]{babel}
\usepackage{amsmath}
\usepackage{mathtools}
\usepackage{hyperref}
\usepackage{xcolor}
\usepackage{color}
\usepackage{ragged2e}
\usepackage{mathrsfs}
\usepackage{csquotes}
\usepackage{listings}
\usepackage[scaled]{beramono}
\usepackage[T1]{fontenc}
\usepackage{matlab-prettifier}
\usepackage{graphicx}
\usepackage{booktabs}

\renewcommand{\indent}{\hspace*{2em}}

% \usepackage{tikz}

% \usetikzlibrary{fit, shapes, arrows}

% \usepackage{courier}
% \usepackage{subfigure}
% \usepackage{enumerate}
% \usepackage{algorithmic}
% \usepackage{algorithm}

% \usepackage{listings}
% \usepackage{lstlinebgrd}

\usetheme{Boadilla}
\usefonttheme[onlymath]{serif}

\newcommand{\matlab}[1]{\lstinline[style=Matlab-editor]!#1!}
\newcommand\blfootnote[1]{%
\begingroup
\renewcommand\thefootnote{}\footnote{#1}%
\addtocounter{footnote}{-1}%
\endgroup
}

\lstset
{
    language = Matlab,
    style = Matlab-editor,
    basicstyle = \mlttfamily\scriptsize,
    escapechar = `,
    numbers = left,
    frame = tb,
}

\lstdefinestyle{output}
{
    language = {},
    basicstyle = \mlttfamily\scriptsize,
    escapechar = `,
    numbers = none,
    showtabs = false,
   	showstringspaces = false,
}

% Sets the templates
\definecolor{navyblue}{RGB}{0, 0, 128}
\definecolor{crimson}{RGB}{128, 16, 0}

\setbeamertemplate{navigation symbols}{}
\setbeamertemplate{headline}{}
\setbeamertemplate{title page}[default][colsep=-4bp,rounded=true]
\setbeamertemplate{footline}[frame number]
\setbeamertemplate{bibliography item}[text]
\setbeamertemplate{theorems}[numbered]

\setbeamercolor{title}{fg=navyblue, bg=white}
\setbeamercolor{frametitle}{fg=navyblue, bg=white}
\setbeamercolor{structure}{fg=navyblue}
\setbeamercolor{button}{fg=white,bg=navyblue}

\setbeamercovered{transparent}

\title{Ecuaciones, propiedades y eliminación}
\subtitle{Modelación de la ingeniería a través de la matemática computacional \\ (TC1003B)}
\author{
    \texorpdfstring{
        \begin{center}
            M.C. Xavier Sánchez Díaz \\
            \href{mailto:sax@tec.mx}{\texttt{sax@tec.mx}}
        \end{center}
    }
    {M.C. Xavier Sánchez Díaz}
}

\institute[Tecnológico de Monterrey]{\includegraphics[scale=0.5]{../img/logo}}
\date{}

\begin{document}

\setlength{\rightskip}{0pt}

\begin{frame}[plain]
    \titlepage        
\end{frame}

\begin{frame}{Outline}
    \tableofcontents
\end{frame}

\section{¿Ecuaciones y matrices?}

\begin{frame}{Ensamblando Robots}{¿Ecuaciones y matrices?}

    \textit{IntelliCorp} produce dos tipos de procesadores, el \texttt{x230} y el \texttt{x260} para sus robots.
    
    Para poder fabricarlos, se necesitan \textbf{silicio}, \textbf{cobre} y \textbf{aluminio}.

    El \texttt{x230} usa 4, 3 y 5 láminas, respectivamente, mientras que el \texttt{x260} usa 5, 2 y 6 placas.

    \begin{center}
        \begin{table}[H]
            \begin{tabular}{@{}ccc@{}}
            \toprule
                                    & x230 & x260 \\ \midrule
            \multicolumn{1}{c|}{Si} & 4    & 5    \\
            \multicolumn{1}{c|}{Cu} & 3    & 2    \\
            \multicolumn{1}{c|}{Al} & 5    & 6    \\ \bottomrule
            \end{tabular}
        \end{table}
    \end{center}
\end{frame}

\begin{frame}{¿Qué tiene más sentido?}{¿Ecuaciones y matrices?}

    ¿Que cada placa se haga con distintos procesadores?

    \begin{align*}
        S & = 4x_1 + 5 x_2 \\
        C & = 3x_1 + 2 x_2 \\
        A & = 5x_1 + 6 x_2
    \end{align*} \pause

    \bigskip

    ¿Que cada procesador se haga con distintas placas?
    
    \begin{align*}
        x_{230} & = 4s + 3c + 5a \\
        x_{260} & = 5s + 2c + 6s
    \end{align*}

\end{frame}

\begin{frame}{Transpuesta}{¿Ecuaciones y matrices?}
    Para que tenga más sentido, podemos \alert{transponer} la matriz.
    Para eso, reescribiremos las \textbf{columnas} de la matriz \textbf{como renglones} y los \textbf{renglones como columnas}:

    \[A = 
        \begin{bmatrix*}
            4 & 5 \\
            3 & 2 \\
            5 & 6
        \end{bmatrix*}
    \]

    \bigskip

    \[A^T = 
        \begin{bmatrix*}
            4 & 3 & 5 \\
            5 & 2 & 6
        \end{bmatrix*}
    \]

\end{frame}

\begin{frame}{De las operaciones al álgebra}{¿Ecuaciones y matrices?}

    Desde ahora, nuestra $A^T$ será $C= \begin{bmatrix*}
        4 & 3 & 5 \\
        5 & 2 & 6
    \end{bmatrix*}$

    \bigskip

    ¿Cuántas placas necesitaríamos para hacer 3 procesadores de cada tipo? \pause

    \bigskip

    \[
        3C = 3 \begin{bmatrix*}
        4 & 3 & 5 \\
        5 & 2 & 6
    \end{bmatrix*} = \begin{bmatrix*}
    12 & 9 & 15 \\
    15 & 6 & 18
\end{bmatrix*} \blacksquare
    \]    
\end{frame}

\begin{frame}{De las operaciones al álgebra}{¿Ecuaciones y matrices?}

    Si la nueva tecnología antiestática utiliza 1 placa adicional de cada material para el \texttt{x230}, y 2 placas de silicio, 1 de cobre y 1 de aluminio adicionales para el \texttt{x260}, ¿cuántas placas necesitaré de ahora en adelante si ahora todos mis procesadores incluirán tecnología antiestática?

    \bigskip \pause

    Primero, ¿cómo es la matriz de costos de la tecnología antiestática? \pause
    
    $$S = \begin{bmatrix*}
        1 & 1 & 1 \\
        2 & 1 & 1
    \end{bmatrix*}$$

    \bigskip \pause

    El nuevo costo entonces es \bigskip

    \[%
        C + S =%
        \begin{bmatrix*}
            4 & 3 & 5 \\
            5 & 2 & 6
        \end{bmatrix*}
        %
        +
        %
        \begin{bmatrix*}
            1 & 1 & 1 \\
            2 & 1 & 1
        \end{bmatrix*}
        %
        =
        %
        \begin{bmatrix*}
            5 & 4 & 6 \\
            7 & 3 & 7
        \end{bmatrix*} \blacksquare
    \]
\end{frame}

\begin{frame}{De las operaciones al álgebra}{¿Ecuaciones y matrices?}
    Nuestra nueva $C$ es ahora
    $C = \begin{bmatrix*}
        5 & 4 & 6 \\
        7 & 3 & 7
    \end{bmatrix*}$.
    Si sabemos que cada placa de silicio cuesta \$4, cada placa de cobre \$2 y cada placa de aluminio \$3, ¿Cuál es el precio total de cada procesador en \$? \pause

    \bigskip

    Nuestro vector de precios es $\mathbf{p} = [4, 2, 3]^T$ así que\dots \pause

    \bigskip

    \begin{align*}
        C\mathbf{p} =
        \begin{bmatrix*}
            5 & 4 & 6 \\
            7 & 3 & 7
        \end{bmatrix*}
        \begin{bmatrix*}
            4 \\ 2 \\ 3    
        \end{bmatrix*} & = 4 \begin{pmatrix*} 5 \\ 7 \end{pmatrix*} + 2 \begin{pmatrix*} 4 \\ 3 \end{pmatrix*} + 3 \begin{pmatrix*} 6 \\ 7 \end{pmatrix*} \\
        & = \begin{bmatrix*} 20 + 8 + 18 \\ 28 + 6 + 21 \end{bmatrix*} \\
        & = \begin{bmatrix*} 46 \\ 55 \end{bmatrix*} \blacksquare
    \end{align*}

\end{frame}

\begin{frame}{De las operaciones al álgebra}{¿Ecuaciones y matrices?}
    Ya sabemos el precio de cada procesador. Ahora queremos saber la resistencia eléctrica de cada uno, así como también su peso. \pause

    La matriz que contiene esta información (la columna de resistencias y la columna de pesos, por cada material) es
    $D = \begin{bmatrix*}
        3 & 2 \\
        1 & 4 \\
        2 & 3
    \end{bmatrix*}$

    \bigskip
    
    ¿Puedo hacer la multiplicación de siempre? ¿Qué matriz obtendré? \pause

    \[
        CD = \begin{bmatrix*}
            31 & 44 \\
            38 & 47
        \end{bmatrix*}
    \]

    Que es la matriz de resistencia y peso (heredados de $D$) de los procesadores (heredados de $C$).
\end{frame}

\begin{frame}{De las operaciones al álgebra}{¿Ecuaciones y matrices?}
    Volvamos a transponer nuestra matriz para poder manejar pedidos \\
    (materiales~$\times$~procesador y procesadores~$\times$~pedido para obtener\\
    materiales $\times$ pedidos):
    $C = C^T = 
    \begin{bmatrix}
        5 & 7 \\
        4 & 3 \\
        6 & 7
    \end{bmatrix}$
    
    \bigskip

    Y asumamos que nos hacen un pedido de 3 procesadores \texttt{x230} y 3 procesadores \texttt{x260}.
    ¿Cuánto material necesito? \pause

    \begin{itemize}[<+->]
        \item ¿Cuál es el vector que representa un pedido de 2 y 0 \texttt{x230} y \texttt{x260} respectivamente?
        \item ¿Y si nuestro pedido fuera de 2 y 2?
        \item ¿Y si fuera de 2 y 3?
        \item ¿Y si mi pedido fuera de 1 y 1?
    \end{itemize}
\end{frame}

\begin{frame}{Matriz escalar}{¿Ecuaciones y matrices?}
    Una \alert{matriz escalar} es una matriz que sólo tiene \textbf{escalares} en la \textbf{diagonal}. Sirven para \textit{escalar} una matriz: cada una columna por un \textit{cierto} factor. \pause

    \bigskip

    ¿Cuál es el resultado de la siguiente operación? \bigskip

    \[
        \begin{bmatrix}
            \color{red} 5 & \color{blue} 7 \\
            \color{red} 4 & \color{blue} 3 \\
            \color{red} 6 & \color{blue} 7
        \end{bmatrix}
        \begin{bmatrix*}
            \color{red} 2 & 0 \\
            0 & \color{blue} 3
        \end{bmatrix*} = \pause
        \begin{bmatrix*}
            \color{red} 10 & \color{blue} 21 \\
            \color{red} 8 & \color{blue} 9 \\
            \color{red} 12 & \color{blue} 21
        \end{bmatrix*}
    \]
\end{frame}

% Los robots
% why is it important
% does it exist in math?
% how to represent it
% how to represent it in matlab
% practical cases

% \section*{Referencias}

% \begin{frame}[t]{Referencias}
    % \nocite{bibID01}
    % \nocite{bibID02}

    % \bibliographystyle{IEEE}
    % \bibliography{biblio}
% \end{frame}

\end{document}