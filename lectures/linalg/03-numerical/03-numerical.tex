\documentclass[spanish, c]{beamer}

\usepackage[utf8]{inputenc}
%\usepackage[spanish, mexico]{babel}
\usepackage{amsmath}
\usepackage{mathtools}
\usepackage{hyperref}
\usepackage{xcolor}
\usepackage{color}
\usepackage{ragged2e}
\usepackage{mathrsfs}
\usepackage{csquotes}
\usepackage{listings}
\usepackage[scaled]{beramono}
\usepackage[T1]{fontenc}
\usepackage{matlab-prettifier}
\usepackage{graphicx}
\usepackage{booktabs}

\renewcommand{\indent}{\hspace*{2em}}

% \usepackage{tikz}

% \usetikzlibrary{fit, shapes, arrows}

% \usepackage{courier}
% \usepackage{subfigure}
% \usepackage{enumerate}
% \usepackage{algorithmic}
% \usepackage{algorithm}

% \usepackage{listings}
% \usepackage{lstlinebgrd}

\usetheme{Boadilla}
\usefonttheme[onlymath]{serif}

\newcommand{\matlab}[1]{\lstinline[style=Matlab-editor]!#1!}
\newcommand\blfootnote[1]{%
\begingroup
\renewcommand\thefootnote{}\footnote{#1}%
\addtocounter{footnote}{-1}%
\endgroup
}

\lstset
{
    language = Matlab,
    style = Matlab-editor,
    basicstyle = \mlttfamily\scriptsize,
    escapechar = `,
    numbers = left,
    frame = tb,
}

\lstdefinestyle{output}
{
    language = {},
    basicstyle = \mlttfamily\scriptsize,
    escapechar = `,
    numbers = none,
    showtabs = false,
   	showstringspaces = false,
}

\makeatletter
\renewcommand*\env@matrix[1][*\c@MaxMatrixCols c]{%
  \hskip -\arraycolsep
  \let\@ifnextchar\new@ifnextchar
  \array{#1}}
\makeatother

% Sets the templates
\definecolor{navyblue}{RGB}{0, 0, 128}
\definecolor{crimson}{RGB}{128, 16, 0}

\setbeamertemplate{navigation symbols}{}
\setbeamertemplate{headline}{}
\setbeamertemplate{title page}[default][colsep=-4bp,rounded=true]
\setbeamertemplate{footline}[frame number]
\setbeamertemplate{bibliography item}[text]
\setbeamertemplate{theorems}[numbered]

\setbeamercolor{title}{fg=navyblue, bg=white}
\setbeamercolor{frametitle}{fg=navyblue, bg=white}
\setbeamercolor{structure}{fg=navyblue}
\setbeamercolor{button}{fg=white,bg=navyblue}

\setbeamercovered{transparent}

\title{Del Álgebra a la Computadora: Métodos Numéricos para Sistemas de Ecuaciones}
\subtitle{Modelación de la ingeniería a través de la matemática computacional \\ (TC1003B)}
\author{
    \texorpdfstring{
        \begin{center}
            M.C. Xavier Sánchez Díaz \\
            \href{mailto:sax@tec.mx}{\texttt{sax@tec.mx}}
        \end{center}
    }
    {M.C. Xavier Sánchez Díaz}
}

\institute[Tecnológico de Monterrey]{\includegraphics[scale=0.5]{../img/logo}}
\date{}

\begin{document}

\setlength{\rightskip}{0pt}

\begin{frame}[plain]
    \titlepage        
\end{frame}

\begin{frame}{Outline}
    \tableofcontents
\end{frame}


% Review on Gauss Elimination
% Gauss Elimination vs Gauss-Jordan vs Montante
% What are we exactly doing when we solve equations?
% The Ax = B form
% Pitfalls of Algebraic forms
% Let's try guessing and approximating
% Errors
% Float Point storage
% Diagonal Matrices

\section{Repaso de Eliminación Gaussiana}

\begin{frame}{Recordando las reglas de oro}{Eliminación}
    Sólo podemos usar cualquiera de las siguientes reglas: \pause
    
    \bigskip

    \begin{enumerate}[<+->]
        \item Intercambiar renglones de lugar: $$R_1 \leftrightarrow R_2$$
        \item Escalar renglones (multiplicando por escalar): $$R_1 \leftarrow cR_1$$
        \item A un renglón existente, restarle un otro renglón escalado: $$R_1 \leftarrow R_1 - cR_2$$
    \end{enumerate}
\end{frame}

\begin{frame}{Otro ejemplo de eliminación}{Eliminación}
    Para el sistema siguiente\dots \pause

    \begin{align*}
         x +  y +  z & = 2 \\
         x      +  z & = 1 \\
        2x + 5y + 2z & = 7 
    \end{align*} \pause

    Consigamos la matriz aumentada del sistema: \\
    $$\begin{bmatrix}[ccc|c]
        1 & 1 & 1 & 2 \\
        1 & 0 & 1 & 1 \\
        2 & 5 & 2 & 7
    \end{bmatrix}$$ \pause

    ¿Qué podemos hacer con ella?
\end{frame}

\begin{frame}{Otro ejemplo de elimnación}{Eliminación}
    Primero reordenemos ($R_2 \longleftrightarrow R_3$) \pause
    $$\begin{bmatrix}[ccc|c]
        1 & 1 & 1 & 2 \\
        2 & 5 & 2 & 7 \\
        1 & 0 & 1 & 1
    \end{bmatrix}$$ \pause

    Y reduzcamos el segundo renglón ($R_2 \gets R_2 - 2R_1$): \pause
    $$\begin{bmatrix}[ccc|c]
        1 & 1 & 1 & 2 \\
        0 & 3 & 0 & 3 \\
        1 & 0 & 1 & 1
    \end{bmatrix}$$ \pause

    Claramente podemos escalar $R_2$: \pause
    $$\begin{bmatrix}[ccc|c]
        1 & 1 & 1 & 2 \\
        0 & 1 & 0 & 1 \\
        1 & 0 & 1 & 1
    \end{bmatrix}$$
\end{frame}

\begin{frame}{Otro ejemplo de elimnación}{Eliminación}
    Podemos reducir ahora las $y$ en $R_1$ ($R_1 \gets R_1 - R_2$): \pause
    
    $$\begin{bmatrix}[ccc|c]
        1 & 0 & 1 & 1 \\
        0 & 1 & 0 & 1 \\
        1 & 0 & 1 & 1
    \end{bmatrix}$$ \pause

    ¿Podemos hacer algo más? \pause ¿Qué significa esta matriz? \pause
    Si la pasamos a ecuaciones\dots \pause
    \begin{align*}
        x + z & = 1 \\
        y & = 1 \\
        x + z & = 1
    \end{align*} \pause
    \dots lo que significa que $y=1, z = 1 - x$ y $x = 1 - z \quad \blacksquare$ \pause

    Este tipo de soluciones \textbf{también son válidas}.

\end{frame}

\section{La forma canónica de los SEL}

\begin{frame}{¿Qué estamos haciendo realmente al resolver un SEL?}{La forma $A\mathbf{x} = \mathbf{b}$}
    Estamos encontrando los valores de $x, y$ y $z$ que hacen que el sistema sea congruente (si es que es posible). \pause

    Asumamos un sistema como el siguiente:
    $$
    \begin{bmatrix}[ccc|c]
        \alert<4>{1} & \alert<4>{0} & \alert<4>{0} & \alert<5>{b_1} \\
        \alert<4>{0} & \alert<4>{1} & \alert<4>{0} & \alert<5>{b_2} \\
        \alert<4>{0} & \alert<4>{0} & \alert<4>{1} & \alert<5>{b_3}
    \end{bmatrix}
    $$ \pause

    ¿Cuál es el valor de $x$? ¿Y el de $y$ y $z$? \pause

    Podemos identificar claramente que a la izquierda conseguimos la \textbf{matriz identidad}\dots \pause
    \dots y a la derecha un \textbf{vector} de valores constantes \pause. ¿A qué nos recuerda esto?
\end{frame}

\begin{frame}{¿Qué estamos haciendo realmente al resolver un SEL?}{La forma $A\mathbf{x} = \mathbf{b}$}
    
    Realmente, podemos \textit{descomponer} el sistema de ecuaciones lineales en tres elementos: \pause

    \bigskip

    \begin{enumerate}[<+->]
        \item Una \textbf{matriz} cuadrada \textbf{de coeficientes} $A = \begin{bmatrix} 1 & 1 & 1 \\ 1 & 0 & 1 \\ 2 & 5 & 2 \end{bmatrix}$
        \item Un \textbf{vector de variables} $\mathbf{x} = \begin{bmatrix}
            x_1 &  x_2 &  x_3\end{bmatrix}^T$
        \item Un \textbf{vector de constantes} $\mathbf{b} = \begin{bmatrix} 2 \\ 1 \\ 7 \end{bmatrix}$
    \end{enumerate}
\end{frame}

\begin{frame}{Reflexión}
    \begin{center}
        \Huge Para cada paso de la eliminación\dots ¿cómo sabes qué paso debería seguir?
    \end{center}
\end{frame}

\begin{frame}{Una idea}
    \begin{center}
        \Huge
        ¿Qué tal si empezamos a poner valores al tanteo, y vamos \textit{calando} distintos valores para cada variable?
    \end{center}
\end{frame}

\begin{frame}{Error}{Errores y representación digital}
Asumamos, pues, que vamos a empezar en un \textbf{valor inicial} y que iremos, poco a poco, variando el valor de cada variable ({\footnotesize duh}) para llegar a una solución aproximada. \pause

En cada paso debo preguntarme: \pause

\begin{itemize}[<+->]
    \item ¿Qué tanto me falta para llegar?
    \item ¿Para dónde debo moverme?
    \item ¿Le sumo o le resto?
\end{itemize} \pause

Después de muchos pasos, yo esperaría estar cada vez más cerca de la solución. ¿Siempre es así?
\end{frame}

\begin{frame}{Error}{Errores y representación digital}
    
    \begin{center}
        {\Huge Story time: \textit{the Wolf}}
    \end{center}
\end{frame}

\begin{frame}{Representación digital}{Errores y representación digital}
    \begin{center}
        {\Huge Story time: \textit{floating points}}
    \end{center}
\end{frame}

\section{Del álgebra a los números}

\begin{frame}{Del álgebra a los números}
    La solemne y civilizada forma \textit{canónica} de un sistema de ecuaciones lineales, la $A\mathbf{x} = \mathbf{b}$ es hermosa, sí; pero no es práctica. \pause
    
    ¿Qué pasa si mi sistema de ecuaciones es de 40 ecuaciones y 40 incógnitas? \pause ¿Y si es de 2000? \pause

    \bigskip

    Dada a la presencia de herramientas para hacer muchos cálculos en poco tiempo, los métodos \textit{iterativos} se volvieron más comunes. \pause
    
    \bigskip

    Es importante entender la diferencia entre la manera \textit{abstracta, perfecta} de resolver \alert{algebraicamente} un sistema de ecuaciones, y la manera \alert{numérica} para hacerlo: usando aproximaciones hasta alcanzar una \textbf{precisión} deseada.
\end{frame}

\begin{frame}{Funciones diferenciables vs Aproximaciones Discretas}
    \begin{center}
        \Huge Mini story time:
        
        \textit{Continuo versus Discreto}
    \end{center}
\end{frame}

% \section*{Referencias}

% \begin{frame}[t]{Referencias}
    % \nocite{bibID01}
    % \nocite{bibID02}

    % \bibliographystyle{IEEE}
    % \bibliography{biblio}
% \end{frame}

\end{document}