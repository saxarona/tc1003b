\documentclass{article}
%\usepackage[a4paper]{geometry}
\usepackage{fullpage}
\usepackage[utf8]{inputenc}
\usepackage[spanish, mexico]{babel}
\usepackage{lipsum}
\usepackage{bm}
\usepackage{upgreek}
\usepackage{enumitem}
\usepackage{mathrsfs}
\usepackage{amsmath}
\usepackage{amssymb}
\usepackage{tikz}
\usepackage{tcolorbox}
\usepackage{csquotes}
\usepackage{helvet}
\usetikzlibrary{arrows, automata}

\tikzset{
    automaton/.style={
        ->, %arrow type
        >=stealth', %arrow head type (bold)
        shorten >=1pt, 
        auto,
        %semithick,
        initial text=$ $, %no start text
    }
}


% mathtools for: Aboxed (put box on last equation in align envirenment)
\usepackage{microtype} %improves the spacing between words and letters

%% COLOR DEFINITIONS

\usepackage{xcolor} % Enabling mixing colors and color's call by 'svgnames'

\definecolor{MyColor1}{rgb}{0.2,0.4,0.6} %mix personal color
\newcommand{\textb}{\color{Black} \usefont{OT1}{lmss}{m}{n}}
\newcommand{\blue}{\color{MyColor1} \usefont{OT1}{lmss}{m}{n}}
\newcommand{\blueb}{\color{MyColor1} \usefont{OT1}{lmss}{b}{n}}
\newcommand{\red}{\color{LightCoral} \usefont{OT1}{lmss}{m}{n}}
\newcommand{\green}{\color{Turquoise} \usefont{OT1}{lmss}{m}{n}}

\DeclareMathOperator{\trace}{trace}
\DeclareMathOperator{\diag}{diag}

%% FONTS AND COLORS

%    SECTIONS

\usepackage{titlesec}
\usepackage{sectsty}
%%%%%%%%%%%%%%%%%%%%%%%%
%set section/subsections HEADINGS font and color
%\sectionfont{\color{Black}}  % sets colour of sections
%\subsectionfont{\color{Black}}  % sets colour of sections

%set section enumerator to arabic number (see footnotes markings alternatives)
\renewcommand\thesection{\arabic{section}} %define sections numbering
\renewcommand\thesubsection{\thesection\arabic{subsection}} %subsec.num.

%define new section style
\newcommand{\mysection}{
\titleformat{\section} [runin] {\usefont{OT1}{lmss}{b}{n}\color{MyColor1}} 
{\thesection} {3pt} {} } 


% %	CAPTIONS
% \usepackage{caption}
% \usepackage{subcaption}
% %%%%%%%%%%%%%%%%%%%%%%%%
% \captionsetup[figure]{labelfont={color=Turquoise}}


%		!!!EQUATION (ARRAY) --> USING ALIGN INSTEAD
%using amsmath package to redefine eq. numeration (1.1, 1.2, ...) 
\renewcommand{\theequation}{\thesection.\arabic{equation}}

\setlength\parindent{0pt}




\makeatletter
\let\reftagform@=\tagform@
\def\tagform@#1{\maketag@@@{(\ignorespaces\textcolor{red}{#1}\unskip\@@italiccorr)}}
\renewcommand{\eqref}[1]{\textup{\reftagform@{\ref{#1}}}}
\makeatother
\usepackage{hyperref}
\hypersetup{colorlinks=true}

% For labeling top of page on every page but first one:
%\usepackage{fancyhdr}

\newcommand{\myclass}{TC1003B -- Modelación de la Ingeniería con Matemática Computacional} % Class name?
\newcommand{\mytitle}{Examen Argumentativo} % Title of document?
\newcommand{\mydate}{11.03.2020} % The date?
\newcommand{\myheader}{
    \begin{flushleft}
        \large
        Nombre: \rule{10 cm}{0.4mm} \hfill Matrícula: \rule{2 cm}{0.4mm}\\[1.5ex]
        Nombre: \rule{10 cm}{0.4mm} \hfill Matrícula: \rule{2 cm}{0.4mm}\\[1.5ex]
        Nombre: \rule{10 cm}{0.4mm} \hfill Matrícula: \rule{2 cm}{0.4mm}\\[1.5ex]
        Nombre: \rule{10 cm}{0.4mm} \hfill Matrícula: \rule{2 cm}{0.4mm}
    \end{flushleft}
}

\title{
    \myclass \\
    \textbf{\mytitle} \\
    % \myheader
    \date{}
}

% You can set the date automatically by replacing "date goes here" with "\today"

% \renewcommand{\rmdefault}{phv} % Arial Font
\renewcommand{\familydefault}{\sfdefault}

% \pagestyle{fancy}
% \fancyhead{}
% \fancyhead[CO,CE]{{\small{{\bf{\mytitle}} -- \myclass}}}

\newcommand{\responserule}{{\large\rule{14 cm}{0.3mm}}}
\newcommand{\shortresponserule}{{\large\rule{5 cm}{0.3mm}}}
\newcommand{\veryshortresponserule}{{\large\rule{3 cm}{0.3mm}}}

\begin{document}
\maketitle

\vspace{-1.5cm}

{%
\footnotesize
\textit{Lee cuidadosamente y contesta lo que se te pide.
Este examen está pensado para resolverse \textbf{de forma individual}.}

\textit{Al momento de contestar, intenten ser lo más explícito posible: se calificarán las competencias con base en lo que esté escrito y sea subido a la plataforma.
Buena suerte.}
}

\section{Lógica}

\subsection{SICT0101B-1/3}

Se tiene un sistema de 3 puertas que tiene un indicador de Hombres, Mujeres, o Niños (hasta 12 años).
Cuando un hombre pasa por la puerta que le corresponde, la puerta emite una señal (que puedes considerar como \textit{verdadera} o 1).
Similarmente, la puerta de mujeres emite una señal cuando una mujer pasa por su puerta, etc.
Utiliza las variables $a, b$ y $c$ como entradas.

\begin{itemize}
    \item Como salida 1 (función 1) un foco se enciende si entra a su puerta un hombre o en su puerta correspondiente una mujer, y no importa si entran niños en su avenida correspondiente.
    (\textit{\footnotesize Usa las correspondientes entradas para que la salida sea 1 o True})
    \item En la salida 2 (función 2) un foco se enciende si entra a su puerta una mujer y (AND) un niño, sin importar si un hombre entra o no.
    \item En la tercera salida (función 3) un foco se enciende si entra (una mujer o (OR) un niño) y no (NOT)  entra un hombre.
\end{itemize}

Nombra tus variables de salida como $f_1, f_2$ y $f_3$ para plantear las funciones de salida llenando primero la tabla de verdad de cada una. Luego, plantea las funciones booleanas.
Obtén la función mínima mediante los teoremas de la lógica booleana.

\section{Conjuntos} 

\subsection{SEG0503A-1/1}

Demuestra que $(X \subseteq A) \wedge (X \subseteq B) \vdash X \subseteq A \cap B$ y da un ejemplo práctico de esta afirmación. (\textit{\footnotesize Hint: empieza asumiendo que algún elemento cualquiera $x$ pertenece a $X$}.)

\subsection{SICT0101B-2/3}

Una empresa embotelladora tiene distintos datos de sus clientes, que pertenecen a distintos sectores económicos.
La base de datos de clientes tiene 4 tablas para los datos:

\begin{itemize}
    \item Ventas
    \item Auditoría
    \item Censo externo
    \item Entorno demográfico
\end{itemize}
Y 4 tablas para los sectores económicos:
\begin{itemize}
    \item Industrias y empresas
    \item Restaurantes
    \item Tiendas de Conveniencia
    \item Casa Habitación
\end{itemize}

El equipo de análisis de datos quiere expresar el resultado de algunas consultas usando \textbf{teoría de conjuntos}.

¿Qué variables utilizarás para denotar cada tabla (conjunto)? Genera y escribe el alfabeto que utilizarás.
Posteriormente, escribe como un conjunto por \textit{descripción} cada una de las siguientes consultas:

\begin{enumerate}
    \item El conjunto de clientes que son tiendas de conveniencia o casa habitación, y que no tienen auditoría
    \item El conjunto de clientes que están mal etiquetados porque salen como tiendas de conveniencia y como restaurantes
    \item El conjunto de clientes que son restaurantes que no tienen censo ni entorno pero sí tienen ventas
    \item El conjunto de clientes que son de cualquier tipo excepto casa habitación
\end{enumerate}

\section{Álgebra Lineal}

\subsection{SICT0101B-3/3, SICT0102A-1/1}

Girls' Generation fue un grupo de K-pop muy famoso entre 2007 y 2017.
El \textit{lineup} original consitía de 9 miembros: Taeyeon, Jessica, Sunny, Tiffany, Hyoyeon, Yuri, Sooyoung, Yoona y Seouhyun, pero Jessica \textit{abandonó} el grupo en 2014.
Si se reunieran todas e hicieran un intercambio de regalos (como en los buenos tiempos de Navidad 2011, cuando salió la canción de Diamond) quedaría una de ellas volando. Para evitar este problema, se sugirió que cada una le diera un regalo a todas las demás.

Usando las distintas herramientas aprendidas en el semestre, contesta:

\begin{itemize}
    \item ¿Cuántos regalos distintos habría en el intercambio?
    \item ¿Cómo sería la diagonal de la matriz de adyacencia del intercambio?
    \item ¿Es la relación reflexiva? ¿Es simétrica? ¿Es transitiva?
    \item ¿Es función?
\end{itemize}


\vfill

\textbf{De acuerdo con el Código de Ética del Tecnológico de Monterrey, mi desempeño en esta actividad estará guiado por la integridad académica.}
\end{document}