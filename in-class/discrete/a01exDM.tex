\documentclass{article}
%\usepackage[a4paper]{geometry}
\usepackage{fullpage}
\usepackage[utf8]{inputenc}
\usepackage[spanish, mexico]{babel}
\usepackage{lipsum}
\usepackage{bm}
\usepackage{upgreek}
\usepackage{enumitem}
\usepackage{mathrsfs}
\usepackage{amsmath}
\usepackage{amssymb}
\usepackage{tikz}
\usepackage{tcolorbox}
\usepackage{csquotes}
\usepackage{helvet}
\usetikzlibrary{arrows, automata}

\tikzset{
    automaton/.style={
        ->, %arrow type
        >=stealth', %arrow head type (bold)
        shorten >=1pt, 
        auto,
        %semithick,
        initial text=$ $, %no start text
    }
}


% mathtools for: Aboxed (put box on last equation in align envirenment)
\usepackage{microtype} %improves the spacing between words and letters

%% COLOR DEFINITIONS

\usepackage{xcolor} % Enabling mixing colors and color's call by 'svgnames'

\definecolor{MyColor1}{rgb}{0.2,0.4,0.6} %mix personal color
\newcommand{\textb}{\color{Black} \usefont{OT1}{lmss}{m}{n}}
\newcommand{\blue}{\color{MyColor1} \usefont{OT1}{lmss}{m}{n}}
\newcommand{\blueb}{\color{MyColor1} \usefont{OT1}{lmss}{b}{n}}
\newcommand{\red}{\color{LightCoral} \usefont{OT1}{lmss}{m}{n}}
\newcommand{\green}{\color{Turquoise} \usefont{OT1}{lmss}{m}{n}}

\DeclareMathOperator{\trace}{trace}
\DeclareMathOperator{\diag}{diag}

%% FONTS AND COLORS

%    SECTIONS

\usepackage{titlesec}
\usepackage{sectsty}
%%%%%%%%%%%%%%%%%%%%%%%%
%set section/subsections HEADINGS font and color
%\sectionfont{\color{Black}}  % sets colour of sections
%\subsectionfont{\color{Black}}  % sets colour of sections

%set section enumerator to arabic number (see footnotes markings alternatives)
\renewcommand\thesection{\arabic{section}} %define sections numbering
\renewcommand\thesubsection{\thesection\arabic{subsection}} %subsec.num.

%define new section style
\newcommand{\mysection}{
\titleformat{\section} [runin] {\usefont{OT1}{lmss}{b}{n}\color{MyColor1}} 
{\thesection} {3pt} {} } 


% %	CAPTIONS
% \usepackage{caption}
% \usepackage{subcaption}
% %%%%%%%%%%%%%%%%%%%%%%%%
% \captionsetup[figure]{labelfont={color=Turquoise}}


%		!!!EQUATION (ARRAY) --> USING ALIGN INSTEAD
%using amsmath package to redefine eq. numeration (1.1, 1.2, ...) 
\renewcommand{\theequation}{\thesection.\arabic{equation}}

\setlength\parindent{0pt}




\makeatletter
\let\reftagform@=\tagform@
\def\tagform@#1{\maketag@@@{(\ignorespaces\textcolor{red}{#1}\unskip\@@italiccorr)}}
\renewcommand{\eqref}[1]{\textup{\reftagform@{\ref{#1}}}}
\makeatother
\usepackage{hyperref}
\hypersetup{colorlinks=true}

% For labeling top of page on every page but first one:
%\usepackage{fancyhdr}

\newcommand{\myclass}{TC1003B -- Modelación de la Ingeniería con Matemática Computacional} % Class name?
\newcommand{\mytitle}{AEX 01 - Actividad Extra 1} % Title of document?
\newcommand{\mydate}{} % The date?
\newcommand{\myheader}{
    \begin{flushleft}
        \large
        Nombre: \rule{10 cm}{0.4mm} \hfill Matrícula: \rule{2 cm}{0.4mm}\\[1.5ex]
        Nombre: \rule{10 cm}{0.4mm} \hfill Matrícula: \rule{2 cm}{0.4mm}\\[1.5ex]
        Nombre: \rule{10 cm}{0.4mm} \hfill Matrícula: \rule{2 cm}{0.4mm}\\[1.5ex]
        Nombre: \rule{10 cm}{0.4mm} \hfill Matrícula: \rule{2 cm}{0.4mm}
    \end{flushleft}
}

\title{
    \myclass \\
    \textbf{\mytitle} \\
    % \myheader
    \date{}
}

% You can set the date automatically by replacing "date goes here" with "\today"

% \renewcommand{\rmdefault}{phv} % Arial Font
\renewcommand{\familydefault}{\sfdefault}

% \pagestyle{fancy}
% \fancyhead{}
% \fancyhead[CO,CE]{{\small{{\bf{\mytitle}} -- \myclass}}}

\newcommand{\responserule}{{\large\rule{14 cm}{0.3mm}}}
\newcommand{\shortresponserule}{{\large\rule{5 cm}{0.3mm}}}
\newcommand{\veryshortresponserule}{{\large\rule{3 cm}{0.3mm}}}

\begin{document}
\maketitle

\vspace{-1.5cm}

{%
\footnotesize
\textit{Esta actividad es una adaptación al examen argumentativo de TC1003B (Modelación de la ingeniería con matemática computacional), que intenta evaluar que los alumnos tengan ciertas competencias en específico.}

\section{Lógica}

\subsection{SICT0101B-1/3}

\textit{\footnotesize Este problema fue generado por el maestro Rafael Dávalos.}

\vspace{2.5ex}

Se tiene un sistema de 3 puertas que tiene un indicador de Hombres, Mujeres, o Niños (hasta 12 años).
Cuando un hombre pasa por la puerta que le corresponde, la puerta emite una señal (que puedes considerar como \textit{verdadera} o 1).
Similarmente, la puerta de mujeres emite una señal cuando una mujer pasa por su puerta, etc.
Utiliza las variables $a, b$ y $c$ como entradas.

\begin{itemize}
    \item Como salida 1 (función 1) un foco se enciende si entra a su puerta un hombre o en su puerta correspondiente una mujer, y no importa si entran niños en su avenida correspondiente.
    (\textit{\footnotesize Usa las correspondientes entradas para que la salida sea 1 o True})
    \item En la salida 2 (función 2) un foco se enciende si entra a su puerta una mujer y (AND) un niño, sin importar si un hombre entra o no.
    \item En la tercera salida (función 3) un foco se enciende si entra (una mujer o (OR) un niño) y no (NOT)  entra un hombre.
\end{itemize}

Nombra tus variables de salida como $f_1, f_2$ y $f_3$ para plantear las funciones de salida llenando primero la tabla de verdad de cada una. Luego, plantea las funciones booleanas.
Obtén la función mínima mediante los teoremas de la lógica booleana.

\pagebreak

\section{Relaciones y Funciones (e intro a grafos)}

\subsection{SICT0101B-3/3, SICT0102A-1/1}

Girls' Generation fue un grupo de K-pop muy famoso entre 2007 y 2017.
El \textit{lineup} original consitía de 9 miembros: Taeyeon, Jessica, Sunny, Tiffany, Hyoyeon, Yuri, Sooyoung, Yoona y Seouhyun, pero Jessica \textit{abandonó} el grupo en 2014.
Si se reunieran todas e hicieran un intercambio de regalos (como en los buenos tiempos de Navidad 2011, cuando salió la canción de Diamond) quedaría una de ellas volando. Para evitar este problema, se sugirió que cada una le diera un regalo a todas las demás.

Usando las distintas herramientas aprendidas en el semestre, contesta:

\begin{itemize}
    \item ¿Cuántos regalos distintos habría en el intercambio?
    \item ¿Cómo sería la diagonal de la matriz de adyacencia del intercambio?
    \item ¿Es la relación reflexiva? ¿Es simétrica? ¿Es transitiva?
    \item ¿Es función?
\end{itemize}

\vspace{8cm}


\section*{Reflexión}
Escribe los conceptos, tips o símbolos que consideres útiles para recordar lo visto en la sesión. Esta hoja te será de utilidad durante el examen.

\vfill

\textbf{De acuerdo con el Código de Ética del Tecnológico de Monterrey, mi desempeño en esta actividad estará guiado por la integridad académica.}
\end{document}