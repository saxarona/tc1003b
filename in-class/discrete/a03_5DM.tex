\documentclass[spanish, 10pt]{article}

\usepackage[table, xcdraw]{xcolor}
\usepackage[utf8]{inputenc}
\usepackage[spanish, mexico]{babel}
\usepackage{helvet}
\usepackage{fullpage}
\usepackage{graphicx}
\usepackage{enumitem}
\usepackage{tikz}
\usepackage{ulem}
\usepackage{url}
\usepackage{hyperref}
\usepackage[margin = 3 cm]{geometry}
\usepackage{amsmath}
\usepackage{amsfonts}
\usepackage{mathrsfs}

\usepackage{matlab-prettifier}
\usepackage{multicol}

\usetikzlibrary{arrows, shapes, trees, calc, decorations.pathreplacing, shapes.misc, positioning, automata}

\setlength\parindent{0pt}

\renewcommand{\familydefault}{\sfdefault}
\newcommand{\responserule}{{\large\rule{14 cm}{0.3mm}}}
\newcommand{\shortresponserule}{{\large\rule{5 cm}{0.3mm}}}
\newcommand{\veryshortresponserule}{{\large\rule{3 cm}{0.3mm}}}
\newcommand{\matlab}[1]{\lstinline[style=Matlab-pyglike]!#1!}


% Specifications for listing package
% \lstset{	
%     basicstyle = \scriptsize\ttfamily,
%     keywordstyle = \color{blue}\ttfamily,
%     stringstyle = \color{red}\ttfamily,
%    	commentstyle = \color{gray}\ttfamily,
%    	tabsize = 3,
%    	breaklines = true,
%    	stepnumber = 1,
%    	showtabs = false,
%    	showstringspaces = false,
%    	frame = none
% }

% Commands for true/false questions
% ----------------------------------------------------------------
\newcommand{\question}[1]{%
	\noindent
	\begin{minipage}[t]{0.15\linewidth}
	\centering		
		\textbf{[\hspace{1 cm}]}
	\end{minipage}%
	\begin{minipage}[t]{0.85\linewidth}
		#1
	\end{minipage}
	\smallskip
}
% ----------------------------------------------------------------

\setlength\parindent{0pt}

\begin{document}

\begin{center}
	{\Large \textbf{Modelación de la ingeniería con matemática computacional (TC-1003B)}}
	
	\bigskip
	{\large \textbf{Actividad 03.5 -- Repaso}}
\end{center}

\bigskip
{\large \textbf{Nombre}: \rule{13.7 cm}{0.4mm}}

% \bigskip
% {\large \textbf{Matrícula}: \rule{5 cm}{0.4mm}}

% \bigskip
% {\large \textbf{Name}: \rule{14 cm}{0.4mm}}

\bigskip
{\large \textbf{Matrícula}: \rule{5 cm}{0.4mm}} %\hfill {\large \textbf{Fecha}: \today}

\bigskip

% {\footnotesize Nota: es probable que esta actividad nos asuste un poco al principio. Es perfectamente normal.
% En efecto, es de mayor dificultad a las que hemos visto anteriormente y probablemente haya dudas.
% Si hay algo que no entiendas, no te quedes sin preguntar.}

\section{Relaciones y funciones}

Sean $A = \{1,2,3,4,5\}$ y $B = \{1,2,3,4,5\}$.
Para las siguientes relaciones, indica si son \textbf{reflexivas}, \textbf{simétricas} o \textbf{transitivas}.
Menciona también si son \textbf{funciones}. En caso de que lo sean, indica si son \textbf{totales} o \textbf{parciales}, y si son \textbf{inyecciones}, \textbf{sobreyecciones} o \textbf{biyecciones}.

\begin{enumerate}[label=\tt \alph*)]
    \itemsep0em
    \item $\{(1,1), (2,2), (3,3), (4,4)\}$
    \item $\{(2,2), (1,1), (3,3), (5,5), (4,4)\}$
    \item $\{(1,2), (2,1), (3,4), (4,3), (3,5), (5,3)\}$
    \item $\{(1,5), (2,3), (3,2), (4,4), (5, 4)\}$
    \item $A \times B$
\end{enumerate}

\vspace{7ex}

\section{Operaciones con conjuntos}

Calcula el resultado de las siguientes operaciones.

\begin{enumerate}[label=\tt \alph*)]
    \itemsep0em
    \item $\{1, 2, 3\} \cup \{z : z \in \mathbb{Z}, 4 \leq z < 10 \}$
    \item $\{1\} \times \{2,3,4\}$
    \item $|\mathscr{P}(\{n : n \in \mathbb{N} \cup \{0\}, n < 15\})|$
    \item $\{10, 20, 30\} \cap \{r : r \in \mathbb{N}\}$
    \item $\{1, 2, 3\} \cap \{4, 5, 6\}$
\end{enumerate}

\pagebreak

\section{Lógica proposicional}

Considera el siguiente alfabeto:

\begin{enumerate}
    \item $P = x$ vale 5
    \item $Q = y$ vale 7
    \item $R = $ la suma de $x$ y $y$ da 12
\end{enumerate}

¿Es cierta la oración $P \wedge Q \implies R$ cuando $P$ y $Q$ son verdaderas y $R$ es falsa? ¿Por qué? Demuéstralo con una tabla de verdad adecuada. Contesta primero las siguientes preguntas:

\begin{itemize}
    \item ¿Cuántas variables tengo?
    \item ¿Cuántos renglones tendrá mi tabla de verdad?
    \item ¿Cuántas columnas \textit{mínimo} necesito en mi tabla de verdad?
\end{itemize}

Ahora escribe tu tabla de verdad con las características necesarias.

Si sabes que $P \wedge Q \implies R$ es cierta, pero $R$ no es cierta, ¿qué puedes \textit{inferir} sobre $P \wedge Q$? Puedes usar tu tabla de verdad para buscar los casos específicos y contestar con esa información.


\vfill

\textbf{Apegándome al Código de Ética de los Estudiantes del Tecnológico de Monterrey, me comprometo a que mi actuación en esta actividad esté regida por la honestidad académica.}

\end{document}