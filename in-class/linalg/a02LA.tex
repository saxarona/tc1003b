\documentclass[spanish, 10pt]{article}

\usepackage[table, xcdraw]{xcolor}
\usepackage[utf8]{inputenc}
\usepackage[spanish, mexico]{babel}
\usepackage{helvet}
\usepackage{fullpage}
\usepackage{graphicx}
\usepackage{enumitem}
\usepackage{tikz}
\usepackage{ulem}
\usepackage{url}
\usepackage{hyperref}
\usepackage[margin = 3 cm]{geometry}
\usepackage{amsmath}
\usepackage{amsfonts}

\usepackage{matlab-prettifier}
\usepackage{multicol}

\usetikzlibrary{arrows, shapes, trees, calc, decorations.pathreplacing, shapes.misc, positioning, automata}

\setlength\parindent{0pt}

\renewcommand{\familydefault}{\sfdefault}
\newcommand{\responserule}{{\large\rule{14 cm}{0.3mm}}}
\newcommand{\shortresponserule}{{\large\rule{5 cm}{0.3mm}}}
\newcommand{\veryshortresponserule}{{\large\rule{3 cm}{0.3mm}}}
\newcommand{\matlab}[1]{\lstinline[style=Matlab-pyglike]!#1!}


% Specifications for listing package
% \lstset{	
%     basicstyle = \scriptsize\ttfamily,
%     keywordstyle = \color{blue}\ttfamily,
%     stringstyle = \color{red}\ttfamily,
%    	commentstyle = \color{gray}\ttfamily,
%    	tabsize = 3,
%    	breaklines = true,
%    	stepnumber = 1,
%    	showtabs = false,
%    	showstringspaces = false,
%    	frame = none
% }

% Commands for true/false questions
% ----------------------------------------------------------------
\newcommand{\question}[1]{%
	\noindent
	\begin{minipage}[t]{0.15\linewidth}
	\centering		
		\textbf{[\hspace{1 cm}]}
	\end{minipage}%
	\begin{minipage}[t]{0.85\linewidth}
		#1
	\end{minipage}
	\smallskip
}
% ----------------------------------------------------------------

\setlength\parindent{0pt}

\begin{document}

\begin{center}
	{\Large \textbf{Modelación de la Ingeniería con Matemática Computacional (TC1003B)}}
	
	\bigskip
	{\large \textbf{Actividad 02 -- Ecuaciones y eliminación}}
\end{center}

\bigskip
{\large \textbf{Nombre}: \rule{13.7 cm}{0.4mm}}

% \bigskip
% {\large \textbf{Matrícula}: \rule{5 cm}{0.4mm}}

% \bigskip
% {\large \textbf{Name}: \rule{14 cm}{0.4mm}}

\bigskip
{\large \textbf{Matrícula}: \rule{5 cm}{0.4mm}} % \hfill {\large \textbf{Fecha}: \today}

\bigskip

{\footnotesize Nota: es probable que esta actividad nos asuste un poco al principio. Es perfectamente normal.
En efecto, es de mayor dificultad a las que hemos visto anteriormente y probablemente haya dudas.
Si hay algo que no entiendas, no te quedes sin preguntar.}

\section{Sistemas de ecuaciones lineales}

\vspace{3ex}

Un \textit{láser combinatorial} es una máquina conceptual que utiliza un espejo esférico y un haz de luz para resolver problemas combinatorios por medio de reflexión óptica.
Para hacerlo, la máquina dispara un láser hacia un punto fijo en el que se sitúa un cristal reflejante.
El cristal puede ser rotado en tres ejes distintos---$x, y $ y $z$.
Para definir la intensidad de la rotación en cada eje se utilizan 3 perillas, cuya influencia sobre cada los ejes depende del cristal en cuestión.

Para el cristal de esta situación problema, se obtuvieron las siguientes tres mediciones:

\begin{itemize}
    \item Si las primeras dos perillas están en $1$, y la tercera perilla está en $2$, la intensidad sobre el eje $x$ es de $8$ unidades.
    \item Si la primera perilla está en $3$, la segunda en $-1$ y la última en $2$, la intensidad sobre el eje $y$ es de $0$ unidades.
    \item Si la primera perilla está en $-1$, la segunda está en $3$ y la última está en $4$, la intensidad sobre el eje $z$ es de $-4$ unidades.
\end{itemize}

Con base en esta información, realiza correctamente lo que se pide:

\begin{enumerate}
    \item ¿Qué representa cada una de las perillas? \hfill \shortresponserule
    \item ¿Y qué representa cada una de las observaciones? \hfill \shortresponserule
    \item Escribe cada observación como una ecuación
    \vspace{11ex}
    \item Escribe la posición de cada una de las perillas en una estructura de datos adecuada
    \vspace{11ex}
    \item Escribe las intensidades sobre cada eje en una estructura de datos adecuada
    \vspace{11ex}
    \item Escribe la estructura de datos resultante si unes las dos estructuras pasadas sobre la dimensión adecuada
    \vspace{11ex}
    \item ¿Qué representa esta información? \hfill \shortresponserule
    \item Obtén los valores de la influencia de cada perilla sobre la intensidad de rotación sobre un eje.
\end{enumerate}

\vspace{10cm}

\section{Reflexión}

Escribe los conceptos, tips o símbolos que consideres útiles para recordar lo visto en la sesión. Esta hoja te será de utilidad durante el examen.

\vfill

\textbf{Apegándome al Código de Ética de los Estudiantes del Tecnológico de Monterrey, me comprometo a que mi actuación en esta actividad esté regida por la honestidad académica.}

\end{document}